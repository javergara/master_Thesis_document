\chapter{Introduction} % Write in your own chapter title
\label{Chapter1}
\lhead{Chapter 1. \emph{Introduction}}

In agriculture there are several environmental threats as drought, pest, and diseases, nutrient deficiency, etc. that affects the crops, reduce their yield and decrease food production. Farmers around the world have the challenge to face those problems, but they do not have enough tools to do it because for years there has been a gap between technology and agriculture. Some few organizations have the power and the technology to make its crops resistant to many threats, but 85\% of the world’s farms are in smallholder farmers hands\cite{nagayets2005small}. Smallholder farmers rely on empirical knowledge, which is less effective in this overcoming farming challenges\cite{hillnhuetter2008early}, technology must be closer to them and enrich their knowledge to improve their food production. 

One good example is Banana, Banana is one of the most important fruit crops in the world  in terms of production volume and trade\cite{faostat2014food}. Only a low percentage of bananas produced are globally traded\cite{lescot2013world}, most of the production is for domestic markets, this crop is a substantial dietary component \cite{abele2007bacterial} due to the nutritional facts and in some countries as Colombia a lot of typical dishes are made from this fruit.
But as mentioned several pests and diseases affect banana crops causing significant yield losses\cite{blomme2017bacterial}, most of these diseases are presented as physical symptoms specially in leaves and if this is not controlled, pests and diseases will affect the whole crop and even the surrounding farms, an early detection system is needed to help farmers to manage those threats. Also, there are many concerns about the change and the increase of those problems due to climate change\cite{garrett2013cambio}\cite{hamada2011impacts}. To help not only banana farmers but all crops and to close the gap between the agriculture and technology, different areas in science are called to create and develop useful and low-cost technological tools applied to agriculture and easy to use for smallholder farmers to help them with all those threats. Some examples are mobile applications \cite{qiang2012mobile} , crop monitoring\cite{berni2009thermal}\cite{hunt2010acquisition}, Internet of thing(IoT)\cite{baoyun2009review} , remote sensing\cite{doraiswamy2003crop} and artificial intelligence \cite{murase2000artificial}. 
Artificial intelligence(AI) and specifically machine learning and deep learning applications are helping in disease detection in humans 
\cite{esteva2017dermatologist}, \cite{rajpurkar2017chexnet} , \cite{shen2017deep}  but also in plants \cite{mohanty2016using} and combined with mobile applications are becoming into great tools in crop diagnosis and management\cite{ramcharan2017deep}. Deep learning models can be trained with different sources of data, but in diseases diagnosis image based\cite{mohanty2016using} models are used since diseases symptoms are clearly observable. The major problems in machine learning and specially in deep learning applications are the source of information, due to these networks require a massive amount of data to be trained and perform a good generalization, is not easy to obtain a good database with reliable and enough data. As disease diagnosis with deep learning use images, there are some traditional methods to do data augmentation such as flipping or mirroring the images \cite{bloice2017augmentor}, but new technologies are coming to generate artificial images with an approach called generative adversarial networks(GAN)\cite{NIPS2014_5423} \cite{perez2017effectiveness} , with these models can be generated artificial data almost real and makes impossible to differentiate between a real and an artificial image for human been\cite{karras2018style}.

Data augmentation with traditional and new techniques are giving solutions to poor data problems and helps to increase the accuracy of the trained models\cite{barbedo2018impact}. To develop this study, context and planning is organized in 8 sections as follows:
\begin{itemize}
\item Section two for the possible title of the final work.
\item Section three and four contents the problem description, the scope of this work within that context and the objectives. 
\item Section five sets the reasons and the importance to conduct this project.
\item Section six is a brief of the state of the art and the theoretical basis for the starting steps.
\item Section seven and eight describes how will be executed this work and establish each step of the process, the activities and the resources needed.
\end{itemize}