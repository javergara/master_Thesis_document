\chapter{Justification} % Write in your own chapter title
\label{Chapter4}
\lhead{Chapter 4. \emph{Justification}}

Applying machine learning techniques such as deep learning and object detection, help farmers not only to diseases classification but detection, for instance, some diseases are presented just in the leaves so this approaches can tell to the farmer which disease and where is it. Fortunately, mayor banana diseases can be treated if are detected in early stages, for instance, mobile applications using deep learning models give a practical tool to the farmers for real in-field detection and additionally crop management advises controlling the current disease. Also, the geolocation information of the images can describe the behavior and the spreading of the diseases through a city or even a country. But the major problem that is not letting this technology going on to the future is the data acquisition. 

Data acquisition for training deep learning models is time-consuming, long duration, experts are required and a massive amount of images with several variations. So data augmentation techniques and artificial data generation as GAN models, have the duty to help in this task making these models more robust and increasing the performance. This study has the purpose of evaluating if GAN models as new artificial data generation technique have the power to generate quality data to take this information into account in deep learning training processes, if GAN models can generate such that quality data more real-world applications in future will be developed due to the capability of new data generation, helping farmers in their daily tasks, such as crop evaluation, providing them useful tools for early diagnosis and disease detection

Farmers have been distant to novel technologies for years, getting those technologies close to them as in this study, reducing the number of images to be acquired to train deep learning model, making the job easier to create useful technology that in the incoming years will contribute to food security.
With the improvement of all these things and technological support can be faced food security problems in the future because with the development of early diagnosis tools and more mobile applications for farmers, they can be lead to have better yield, improve their food production and raise their gains warranting the local food needs. Giving them technological support to face those problems and to improve their crops, making them closer to new technologies and creating tools is an excellent way to make the world more sustainable.